\documentclass[twocolumn]{aastex62}
\usepackage{natbib}
\usepackage[T1]{fontenc}

\let\oldAA\AA
\renewcommand{\AA}{\text{\normalfont\oldAA}}

\usepackage{hyperref}
\usepackage[capitalise]{cleveref}

\newcommand{\vdag}{(v)^\dagger}
\newcommand\aastex{AAS\TeX}
\newcommand\latex{La\TeX}

\received{June 8, 2020}

\submitjournal{ApJ}
\shorttitle{Selection and AGN variability}
\shortauthors{Suberlak et al.}

\newcommand{\project}[1]{\textsf{#1}}

\begin{document}

\title{The impact of selection on perceived long-term AGN variability}

\correspondingauthor{Krzysztof Suberlak}
\email{suberlak@uw.edu}

\author[0000-0002-9589-1306]{Krzysztof L. Suberlak}
\affiliation{Department of Astronomy, University of Washington, Box 351580, Seattle, WA 98195, USA}


\author[0000-0001-5250-2633]{\v{Z}eljko Ivezi\'c}
\affiliation{Department of Astronomy, University of Washington, Box 351580, Seattle, WA 98195, USA}


\begin{abstract}

We use the Sloan Digital Sky Survey (SDSS) Stripe 82 quasar light curves to investigate a claim by \citet{caplar2020} that quasars on average faded between SDSS and HSC observations.

\end{abstract}


\section{Introduction}
\label{sec:introduction}
%
%
%
\section{Methods: simulation setup}
\label{sec:methods}

We use the simulated LSST quasar catalog (simQSO v.1.1 prepared for data challenge 3 - DC3b). The bolometric luminosity and redshift distribution were generated by Mislav Balokovi{\'c} using Bongiorno et al. luminosity function for $M_{Bol} < -15$ and over an area of 100 deg$^{2}$ - the spatial locations were randomly drawn from $0<ra<10$ and $-5<\delta<+5$. The distance modulus, $DM$, corresponds to the standard cosmology: $\Omega_{m}, \Omega_{l}, h =0.27, 0.73, 0.71$ . The LSST magnitudes, and rest-frame $M_{i}$ (based on $M_{Bol}$), were computed by Lynne Jones for the van den Berk composite quasar  spectrum for the spectral energy distribution slope alpha=0. The black hole masses ($M_{BH}$) were generated using Chelsea MacLeod's best fits ($\log_{10}{(M_{BH})} = 2.4 - 0.25 M_{i}$), with a random scatter of 0.5 dex.

To reduce the number of simulated light curves, we first select  69,152 of 1 mln quasars brighter than $i<24$ (which is several mangitudes above the 5$\sigma$ limit for HSC or SDSS). We use the following relationship between MBH, Mi and the Damped Random Walk parameters $\tau$, $SF_{\infty}$ derived in \cite{macleod2010}:

\begin{eqnarray}
\label{eq:powlawmodel}
\log_{10}{f} = &A& + B \log_{10}\left( \lambda_{RF} / 4000 \mbox{\AA} \right) + C (M_{i} + 23) \nonumber \\
&+& D \log_{10}{\left( M_{BH} / 10^{9} M_{\odot}  \right)} 
\end{eqnarray} 

where  $\lambda_{RF} = \lambda_{OBS} / (1+z)$, and we assume that for LSST r-band $\lambda_{OBS} =6204 \mbox{\AA} $, and we use the 
 improved fit coefficients from Suberlak et al, in prep. ($f=\tau$, A,B,C,D=2.597,0.17,0.035,0.141, and for $f=SF_{\infty}$, -0.476 -0.479,0.118,0.118, respectively, derived using SDSS-PS1 r-band).

To explore a similar baseline as the SDSS-HSC observations of \citet{caplar2020}, we simulate 20-year, well-sampled (1000 points) light curves, with uniform cadence. We use the Damped Random Walk model \citep{kelly2009}, where at each timestep a point is drawn from a Gaussian distribution, with mean and variance re-calculated according to the \citet{macleod2010} prescription (also see Suberlak et al. in prep for details).  To improve the statistics, for each seed simQSO quasar we run 10 realizations of the DRW model (which produces light curves with different stochastic behavior, but identical  $\tau$, $SF_{\infty}$, seed mean r-band magnitude). This produces in 691,520 quasar light curves. We use the magnitude-dependent LSST photometric uncertainty model, adding 0.03 mag in quadrature to simulate the SDSS uncertainty level. To simulate observational conditions to each epoch we add a heteroscedastic Gaussian noise. 


To emulate \cite{caplar2020} results we consider the statistics considering the first $r(t=0 yr)$  (SDSS-like) and last  $r(t=20 yr)$ (HSC-like) epochs. To increase the magnitude of the effect caused by quasar variability, we also simulated light curves with $SF_{\infty}$ increased by $f_{SF}=1.25,1.5,2$.  

To show what is the impact of magnitude cut, we select $r(t=20 yr)<19$, which for $f_{SF}=1.0$ admits 9557 objects, and a control sample of the same number of objects chosen randomly from the parent distribution. We plot on Fig.~\ref{fig:dmag_redshift} the median magnitude difference $\Delta = r(t=20 yr)  - r(t=0 yr)$ as a function of quasar redshift. For $f_{SF}=1.0$ the difference between magnitude-limited and control sample is below 2$\sigma$ level (barely noticetable, considerable noise). For increased quasar variability $f_{SF}=1.5$ (Fig.~\ref{fig:dmag_redshift_inc}) the distinction is more noticeable, but there is no redshift trend as on Fig.1 of \citet{caplar2020}. Nearer quasars (at lower redshift) have higher variability ($SF_{\infty}$) as shown on Fig.~\ref{fig:sf_redshift}. It is the result of the initial correlation between $M_{i}$ and redshift (Figs.~\ref{fig:mag_redshift_sfinf_24},~\ref{fig:mag_redshift_sfinf_20})



\begin{figure*}  % code/analyzeQSOcatalog.ipynb
	\plotone{dmag_z_1-0_SF}
	\caption{The main panel depicts the median difference $\Delta m$ in r-band magnitude between first (0 yr) epoch, and last (20 yr) epoch, plotted as a function of redshift, for 9557 objects with $r(t=20 yr)<19$ (blue), and equal number of objects chosen at random (orange). The right hand side panel is the histogram of$\Delta m$ as well as median for both samples. } 
	\label{fig:dmag_redshift}
\end{figure*}
 

\begin{figure*}  % code/analyzeQSOcatalog.ipynb
	\plotone{dmag_z_1-5_SF}
	\caption{Same as Fig.~\ref{fig:dmag_redshift}, but with the derived $SF_{\infty}$ augmented by a factor of 1.5 to increase the amplitude of quasar variability. The difference between magnitude-limited (blue) and random (orange) sample is more pronounced.} 
	\label{fig:dmag_redshift_inc}
\end{figure*}

\begin{figure*}  % code/analyzeQSOcatalog.ipynb
	\plotone{sf_vs_z_1-0_SF.png}
	\caption{The median asymptotic structure function ($SF_{\infty}$) as a function of redshift. The decreasing trend is a result of the magnitude-truncated distribution, as illustrated on Figs.~\ref{fig:mag_redshift_sfinf_24} and ~\ref{fig:mag_redshift_sfinf_20}.} 
	\label{fig:sf_redshift}
\end{figure*}



\begin{figure*}  % code/analyzeQSOcatalog.ipynb
	\plotone{mag_redshift_sfinf_r_lt_24.png}
	\caption{The redshift distribution of absolute i-band magnitude $M_{i}$ for simQSO (left panel). The right panel depicts the median $SF_{\infty}$ in each bin.} 
	\label{fig:mag_redshift_sfinf_24}
\end{figure*}

\begin{figure*}  % code/analyzeQSOcatalog.ipynb
	\plotone{mag_redshift_sfinf_r_lt_20.png}
	\caption{Same as Fig.~\ref{fig:mag_redshift_sfinf_19}, but for $r < 20$, showing the magnitude-limited sample. } 
	\label{fig:mag_redshift_sfinf_20}
\end{figure*}



\section{Results}



\section{Summary and Discussion}
 


%%%%%%%%%%%%%%%%%%%%%%%%%%%%%%%%%%%%%%%%%%%%%%%%%%

%%%%%%%%%%%%%%%%%%%% REFERENCES %%%%%%%%%%%%%%%%%%

% The best way to enter references is to use BibTeX:

\bibliographystyle{aasjournal} 
\bibliography{references}

%%%%%%%%%%%%%%%%%%%%%%%%%%%%%%%%%%%%%%%%%%%%%%%%%%

\end{document}

