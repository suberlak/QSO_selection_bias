\documentclass[twocolumn]{aastex62}
\usepackage{natbib}
\usepackage[T1]{fontenc}

\let\oldAA\AA
\renewcommand{\AA}{\text{\normalfont\oldAA}}

\usepackage{hyperref}
\usepackage[capitalise]{cleveref}

\newcommand{\vdag}{(v)^\dagger}
\newcommand\aastex{AAS\TeX}
\newcommand\latex{La\TeX}

\received{June 8, 2020}

\submitjournal{ApJ}
\shorttitle{Selection and AGN variability}
\shortauthors{Suberlak et al.}

\newcommand{\project}[1]{\textsf{#1}}

\begin{document}

\title{The impact of selection on perceived long-term AGN variability}

\correspondingauthor{Krzysztof Suberlak}
\email{suberlak@uw.edu}

\author[0000-0002-9589-1306]{Krzysztof L. Suberlak}
\affiliation{Department of Astronomy, University of Washington, Box 351580, Seattle, WA 98195, USA}


\author[0000-0001-5250-2633]{\v{Z}eljko Ivezi\'c}
\affiliation{Department of Astronomy, University of Washington, Box 351580, Seattle, WA 98195, USA}


\begin{abstract}

We use the Sloan Digital Sky Survey (SDSS) Stripe 82 quasar light curves to investigate a claim by \citet{caplar2020} that quasars on average faded between SDSS and HSC observations.

\end{abstract}


\section{Introduction}
\label{sec:introduction}
%
%
%
\section{Methods: simulation setup}
\label{sec:methods}

\begin{figure*}  % code/PLOT_Fig2_George_Celerite.ipynb
	\plotone{fig1}
	\caption{figure caption} 
	\label{fig:}
\end{figure*}
 

\section{Results}



\section{Summary and Discussion}
 


%%%%%%%%%%%%%%%%%%%%%%%%%%%%%%%%%%%%%%%%%%%%%%%%%%

%%%%%%%%%%%%%%%%%%%% REFERENCES %%%%%%%%%%%%%%%%%%

% The best way to enter references is to use BibTeX:

\bibliographystyle{aasjournal} 
\bibliography{references}

%%%%%%%%%%%%%%%%%%%%%%%%%%%%%%%%%%%%%%%%%%%%%%%%%%

\end{document}

